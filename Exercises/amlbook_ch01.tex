\documentclass[12pt,letterpaper]{article}

\usepackage[english]{babel}
\usepackage[utf8]{inputenc}
\usepackage[T1]{fontenc}

\usepackage{fullpage}
\usepackage[top=2cm, bottom=4.5cm, left=2.5cm, right=2.5cm]{geometry}
\usepackage{amsmath,amsthm,amsfonts,amssymb,amscd}
\usepackage{lastpage}
\usepackage{enumerate}
\usepackage{fancyhdr}
\usepackage{mathrsfs}
\usepackage{xcolor}
\usepackage{graphicx}
\usepackage{listings}
\usepackage{hyperref}

\hypersetup{%
  colorlinks=true,
  linkcolor=blue,
  linkbordercolor={0 0 1}
}
\newcommand{\bx}{\boldsymbol{x}}
\newcommand{\bw}{\boldsymbol{w}}
\newcommand{\sign}{\operatorname{sign}}
\renewcommand\lstlistingname{Algorithm}
\renewcommand\lstlistlistingname{Algorithms}
\def\lstlistingautorefname{Alg.}
%\setlength{\parindent}{0cm}
\lstdefinestyle{Python}{
    language        = Python,
    frame           = lines, 
    basicstyle      = \footnotesize,
    keywordstyle    = \color{blue},
    stringstyle     = \color{green},
    commentstyle    = \color{red}\ttfamily
}

\setlength{\parindent}{0.0in}
\setlength{\parskip}{0.05in}

% Edit these as appropriate
\newcommand\course{Rener Oliveira}
\newcommand\hwnumber{1}                  % <-- homework number
\newcommand\NetIDa{netid19823}           % <-- NetID of person #1
\newcommand\NetIDb{netid12038}           % <-- NetID of person #2 (Comment this line out for problem sets)

\pagestyle{fancyplain}
\headheight 35pt              % <-- Comment this line out for problem sets (make sure you are person #1)
\lhead{EMAp FGV}
\chead{\textbf{\Large List 1 \\ Machine Learning}}
\rhead{\small{\course \\ \today}}
\lfoot{}
\cfoot{}
\rfoot{\small\thepage}
\headsep 1.5em

\begin{document}
	\textbf{Exercise 1.3:}\cite{yaser2012learning} 
	
	The weight update rule in (1.3) has the nice interpretation that it moves in the direction of classifying $\bx(t)$ correctly. 
	\begin{enumerate}[(a)]
		\item Show that $y(t)\bw^{T}(t)\bx(t)<0.$ \emph{[Hint: $\bx(t)$ is misclassified by $\bw(t)$.]}
		\subitem \textit{Solution:}
		Since $\bx(t)$ is misclassified by $\bw(t)$, we have that $y(t)\neq\sign(\bw^{T}(t)\bx(t))$, so $y(t)$ has the opposite sign of $\bw^{T}(t)\bx(t)$, which implies directly that $y(t)\bw^{T}(t)\bx(t)<0.$
		
		\item Show that $y(t)\bw^{T}(t+1)\bx(t)>y(t)\bw^{T}(t)x(t)$. \emph{[Hint: Use (1.3).]}
		
		\subitem \textit{Solution:}
			Here's the update rule (1.3): 
			
			$$\bw(t+1)=\bw(t)+y(t)\bx(t)$$
			
			Using it, we get:
			
			\begin{align*}
				 &y(t)\bw^{T}(t+1)\bx(t)\\
				=&y(t)[\bw(t)+y(t)\bx(t)]^{T}\bx(t)\\
				=&y(t)[\bw^{T}(t)+y(t)\bx^{T}(t)]\bx(t)\\
				=&y(t)\bw^{T}(t)\bx(t)+y^2(t)\bx^{T}(t)\bx(t)\\
				=&y(t)\bw^{T}(t)\bx(t)+y^2(t)\langle\bx(t),\bx(t)\rangle\\
				=&y(t)\bw^{T}(t)\bx(t)+y^2(t)||\bx(t)||_2^2,
			\end{align*}
			
			and since $\forall ~t=0,1,2,...$ we have $y^2(t)||\bx(t)||_2^2>0$, concluding that $y(t)\bw^{T}(t+1)\bx(t)>y(t)\bw^{T}(t)x(t)$. One could argue that in the case where $\bx$ is a null vector, an equality would hold instead of inequality, but since que have fixed $x_0(t)=1$ $\forall ~t$, such a case is impossible.
		
		\item As far as classifying $\bx(t)$ is concerned, argue that the move from $\bw(t)$ to $\bw(t+1)$ is a move ``in the right direction''.
		
		\subitem \textbf{Solution:} Assume that $\bx(t)$ is misclassified by $\bw(t)$ and that $y(t)=1$. We want to prove that $w(t+1)$ moves the boundary ``in the right direction'', that is, $\bw^{T}(t+1)\bx(t)$ is a strictly increasing function of $t$. If such a thing is true, eventually we would reach a $t$ such that $\bw^{T}(t+1)\bx(t)>0$, which gonna classify $\bx(t)$ correctly.
		
		Such monotone property is already proved in the previous item. Because, if $y(t)=1$, by (b) we have $\bw^{T}(t+1)\bx(t)>\bw^{T}(t)x(t)$.
		
		In the case where $y(t)=-1$, and $\sign(\bx^{T}(t)\bx(t))=1$ (misclassified), we would want $\bw^{T}(t+1)\bx(t)$ to be a strictly decreasing funtion of $t$, which by (b) is true, since $y(t)=-1$ implies $\bw^{T}(t+1)\bx(t)<\bw^{T}(t)x(t)$
		
		\textbf{}		 
	\end{enumerate}
	
\newpage

% \addcontentsline{toc}{section}{Referências}
\bibliographystyle{plain}
\bibliography{references}
\end{document}